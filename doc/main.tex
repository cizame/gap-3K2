% generated by GAPDoc2LaTeX from XML source (Frank Luebeck)
\documentclass[a4paper,11pt]{report}

\usepackage{a4wide}
\sloppy
\pagestyle{myheadings}
\usepackage{amssymb}
\usepackage[latin1]{inputenc}
\usepackage{makeidx}
\makeindex
\usepackage{color}
\definecolor{FireBrick}{rgb}{0.5812,0.0074,0.0083}
\definecolor{RoyalBlue}{rgb}{0.0236,0.0894,0.6179}
\definecolor{RoyalGreen}{rgb}{0.0236,0.6179,0.0894}
\definecolor{RoyalRed}{rgb}{0.6179,0.0236,0.0894}
\definecolor{LightBlue}{rgb}{0.8544,0.9511,1.0000}
\definecolor{Black}{rgb}{0.0,0.0,0.0}

\definecolor{linkColor}{rgb}{0.0,0.0,0.554}
\definecolor{citeColor}{rgb}{0.0,0.0,0.554}
\definecolor{fileColor}{rgb}{0.0,0.0,0.554}
\definecolor{urlColor}{rgb}{0.0,0.0,0.554}
\definecolor{promptColor}{rgb}{0.0,0.0,0.589}
\definecolor{brkpromptColor}{rgb}{0.589,0.0,0.0}
\definecolor{gapinputColor}{rgb}{0.589,0.0,0.0}
\definecolor{gapoutputColor}{rgb}{0.0,0.0,0.0}

%%  for a long time these were red and blue by default,
%%  now black, but keep variables to overwrite
\definecolor{FuncColor}{rgb}{0.0,0.0,0.0}
%% strange name because of pdflatex bug:
\definecolor{Chapter }{rgb}{0.0,0.0,0.0}
\definecolor{DarkOlive}{rgb}{0.1047,0.2412,0.0064}


\usepackage{fancyvrb}

\usepackage{mathptmx,helvet}
\usepackage[T1]{fontenc}
\usepackage{textcomp}


\usepackage[
            pdftex=true,
            bookmarks=true,        
            a4paper=true,
            pdftitle={Written with GAPDoc},
            pdfcreator={LaTeX with hyperref package / GAPDoc},
            colorlinks=true,
            backref=page,
            breaklinks=true,
            linkcolor=linkColor,
            citecolor=citeColor,
            filecolor=fileColor,
            urlcolor=urlColor,
            pdfpagemode={UseNone}, 
           ]{hyperref}

\newcommand{\maintitlesize}{\fontsize{50}{55}\selectfont}

% write page numbers to a .pnr log file for online help
\newwrite\pagenrlog
\immediate\openout\pagenrlog =\jobname.pnr
\immediate\write\pagenrlog{PAGENRS := [}
\newcommand{\logpage}[1]{\protect\write\pagenrlog{#1, \thepage,}}
%% were never documented, give conflicts with some additional packages

\newcommand{\GAP}{\textsf{GAP}}

%% nicer description environments, allows long labels
\usepackage{enumitem}
\setdescription{style=nextline}

%% depth of toc
\setcounter{tocdepth}{1}





%% command for ColorPrompt style examples
\newcommand{\gapprompt}[1]{\color{promptColor}{\bfseries #1}}
\newcommand{\gapbrkprompt}[1]{\color{brkpromptColor}{\bfseries #1}}
\newcommand{\gapinput}[1]{\color{gapinputColor}{#1}}


\begin{document}

\logpage{[ 0, 0, 0 ]}
\begin{titlepage}
\mbox{}\vfill

\begin{center}{\maintitlesize \textbf{\textsf{3K2}\mbox{}}}\\
\vfill

\hypersetup{pdftitle=\textsf{3K2}}
\markright{\scriptsize \mbox{}\hfill \textsf{3K2} \hfill\mbox{}}
{\Huge \textbf{Cayley cages algorithms\mbox{}}}\\
\vfill

{\Huge Version 1.0\mbox{}}\\[1cm]
{16 February 2016\mbox{}}\\[1cm]
\mbox{}\\[2cm]
{\Large \textbf{ Rafael Villarroel-Flores   \mbox{}}}\\
{\Large \textbf{Citlalli Zamora-Mej{\a'\i}a  \mbox{}}}\\
\hypersetup{pdfauthor= Rafael Villarroel-Flores   ; Citlalli Zamora-Mej{\a'\i}a  }
\end{center}\vfill

\mbox{}\\
{\mbox{}\\
\small \noindent \textbf{ Rafael Villarroel-Flores   }  Email: \href{mailto://rvf0068@gmail.com} {\texttt{rvf0068@gmail.com}}\\
  Homepage: \href{http://rvf0068.github.io} {\texttt{http://rvf0068.github.io}}}\\
{\mbox{}\\
\small \noindent \textbf{Citlalli Zamora-Mej{\a'\i}a  }  Email: \href{mailto://cizame@gmail.com} {\texttt{cizame@gmail.com}}}\\
\end{titlepage}

\newpage\setcounter{page}{2}
{\small 
\section*{Copyright}
\logpage{[ 0, 0, 1 ]}
 {\copyright} 2016 by Rafael Villarroel-Flores and Citlalli Zamora-Mej{\a'\i}a

 \textsf{3K2} package is free software; you can redistribute it and/or modify it under the
terms of the \href{http://www.fsf.org/licenses/gpl.html} {GNU General Public License} as published by the Free Software Foundation; either version 2 of the License,
or (at your option) any later version. \mbox{}}\\[1cm]
\newpage

\def\contentsname{Contents\logpage{[ 0, 0, 2 ]}}

\tableofcontents
\newpage

 
\chapter{\textcolor{Chapter }{Gr{\a'a}ficas localmente 3K2}}\label{Gráficas localmente 3K2}
\logpage{[ 1, 0, 0 ]}
\hyperdef{L}{X7D713E817FA8487B}{}
{
  
\section{\textcolor{Chapter }{Gr{\a'a}ficas de Cayley localmente 3K2}}\label{Gráficas de Cayley localmente 3K2}
\logpage{[ 1, 1, 0 ]}
\hyperdef{L}{X7C7DF3F586E8A8D5}{}
{
  

\subsection{\textcolor{Chapter }{CCEliminaInversos}}
\logpage{[ 1, 1, 1 ]}\nobreak
\hyperdef{L}{X78B540B77A6AD15A}{}
{\noindent\textcolor{FuncColor}{$\triangleright$\ \ \texttt{CCEliminaInversos({\mdseries\slshape lista})\index{CCEliminaInversos@\texttt{CCEliminaInversos}}
\label{CCEliminaInversos}
}\hfill{\scriptsize (function)}}\\


 La funci{\a'o}n requiere una lista de elementos de un grupo y regresa la lista
sin inversos. }

 

\subsection{\textcolor{Chapter }{CCConjuntoT1}}
\logpage{[ 1, 1, 2 ]}\nobreak
\hyperdef{L}{X85482E4785A540C5}{}
{\noindent\textcolor{FuncColor}{$\triangleright$\ \ \texttt{CCConjuntoT1({\mdseries\slshape elemento, elemento, elemento})\index{CCConjuntoT1@\texttt{CCConjuntoT1}}
\label{CCConjuntoT1}
}\hfill{\scriptsize (function)}}\\


 Requiere tres elementos de un mismo grupo $a$, $b$ y $c$. Verifica que estos elementos cumplan con las condiciones necesarias para
crear una gr{\a'a}fica de Cayley localmente $3K_2$ del tipo uno. En caso de cumplir las condiciones regresa la lista de seis
elementos $[a,a^{-1},b,b^{-1},c,c^{-1}]$, de lo contrario regresa fail. }

 

\subsection{\textcolor{Chapter }{CCConjuntoT2}}
\logpage{[ 1, 1, 3 ]}\nobreak
\hyperdef{L}{X7CD8BB5C82DCADEA}{}
{\noindent\textcolor{FuncColor}{$\triangleright$\ \ \texttt{CCConjuntoT2({\mdseries\slshape elemento, elemento})\index{CCConjuntoT2@\texttt{CCConjuntoT2}}
\label{CCConjuntoT2}
}\hfill{\scriptsize (function)}}\\


 Requiere dos elementos de un mismo grupo $a$ y $b$. Verifica que estos elementos cumplan con las condiciones necesarias para
crear una gr{\a'a}fica de Cayley localmente $3K_2$ del tipo dos. En caso de cumplir las condiciones regresa la lista de seis
elementos $[a,a^{-1},b,b^{-1},a^{-1}b,b^{-1}a]$, de lo contrario regresa fail. }

 

\subsection{\textcolor{Chapter }{CCCantidadDeGrupos}}
\logpage{[ 1, 1, 4 ]}\nobreak
\hyperdef{L}{X8534B069792D4040}{}
{\noindent\textcolor{FuncColor}{$\triangleright$\ \ \texttt{CCCantidadDeGrupos({\mdseries\slshape n{\a'u}mero, n{\a'u}mero})\index{CCCantidadDeGrupos@\texttt{CCCantidadDeGrupos}}
\label{CCCantidadDeGrupos}
}\hfill{\scriptsize (function)}}\\


 Recibe dos n{\a'u}meros naturales, los que se interpretan como un intervalo en
el cual se desea saber la cantidad de grupos de orden $i$ con $i \in [a,b]$, para cada $i$. }

 

\subsection{\textcolor{Chapter }{CCPosibleCuello}}
\logpage{[ 1, 1, 5 ]}\nobreak
\hyperdef{L}{X7F27D73C7A1B3AC0}{}
{\noindent\textcolor{FuncColor}{$\triangleright$\ \ \texttt{CCPosibleCuello({\mdseries\slshape lista})\index{CCPosibleCuello@\texttt{CCPosibleCuello}}
\label{CCPosibleCuello}
}\hfill{\scriptsize (function)}}\\


 Recibe una lista $T$ de seis elementos de un grupo; $T$ es un conjunto que genera una gr{\a'a}fica de Cayley localmente $3K_2$. La funci{\a'o}n revisa cuales son los dos tama{\~n}os posibles de "cuello de
tri{\a'a}ngulos" de la gr{\a'a}fia de Cayley generada por $T$ y lo reporta. }

 

\subsection{\textcolor{Chapter }{CCPosiblesT}}
\logpage{[ 1, 1, 6 ]}\nobreak
\hyperdef{L}{X7CA39B2D85FE2920}{}
{\noindent\textcolor{FuncColor}{$\triangleright$\ \ \texttt{CCPosiblesT({\mdseries\slshape lista, n{\a'u}mero{\textunderscore}1{\textunderscore}o{\textunderscore}2})\index{CCPosiblesT@\texttt{CCPosiblesT}}
\label{CCPosiblesT}
}\hfill{\scriptsize (function)}}\\


 Recibe dos argumentos, el primero es una lista de elementos de un grupo, el
segundo es el n{\a'u}mero uno o dos, seg{\a'u}n la candici{\a'o}n para crear
gr{\a'a}ficas de Cayley localmente $3K_{2}$ que se quiera verificar. La funci{\a'o}n regresa una nueva lista donde cada
entrada contiene dos o tres elementos, seg{\a'u}n sea el caso, de la lista
original. }

 

\subsection{\textcolor{Chapter }{CCEsGraficaDeCayley}}
\logpage{[ 1, 1, 7 ]}\nobreak
\hyperdef{L}{X829E31E1812E7804}{}
{\noindent\textcolor{FuncColor}{$\triangleright$\ \ \texttt{CCEsGraficaDeCayley({\mdseries\slshape gr{\a'a}fica})\index{CCEsGraficaDeCayley@\texttt{CCEsGraficaDeCayley}}
\label{CCEsGraficaDeCayley}
}\hfill{\scriptsize (function)}}\\


 Recibe una gr{\a'a}fica $G$ y verifica si $G$ es de Cayley o no, en caso afirmativo regresa true o de lo contrario false. }

 

\subsection{\textcolor{Chapter }{CCListaTBuenas}}
\logpage{[ 1, 1, 8 ]}\nobreak
\hyperdef{L}{X7EFAA0237E0004DC}{}
{\noindent\textcolor{FuncColor}{$\triangleright$\ \ \texttt{CCListaTBuenas({\mdseries\slshape grupo, n{\a'u}mero{\textunderscore}1{\textunderscore}o{\textunderscore}2})\index{CCListaTBuenas@\texttt{CCListaTBuenas}}
\label{CCListaTBuenas}
}\hfill{\scriptsize (function)}}\\


 Recibe dos parametros el primero es un grupo y el segundo es el n{\a'u}mero
uno o dos, seg{\a'u}n sea el caso de las condiciones para construir
gr{\a'a}ficas de Cayley localmente $3K_2$. La funci{\a'o}n regresa una lista, cuyos elementos son conjuntos de seis
elementos del grupo dado, los cuales construyen gr{\a'a}ficas de Cayley
localmente $3K_2$. }

 

\subsection{\textcolor{Chapter }{CCExaminaGrupo}}
\logpage{[ 1, 1, 9 ]}\nobreak
\hyperdef{L}{X828D9B4C8225E953}{}
{\noindent\textcolor{FuncColor}{$\triangleright$\ \ \texttt{CCExaminaGrupo({\mdseries\slshape grupo, n{\a'u}mero{\textunderscore}1{\textunderscore}o{\textunderscore}2})\index{CCExaminaGrupo@\texttt{CCExaminaGrupo}}
\label{CCExaminaGrupo}
}\hfill{\scriptsize (function)}}\\


 Recibe tres argumentos el primero es un grupo, el segundo es el tama{\~n}o del
cuello de tri{\a'a}ngulos que se desea y el tercero es el n{\a'u}mero uno o
dos seg{\a'u}n sea el caso de acuerdo al tipo, uno o dos, de gr{\a'a}ficas de
Cayley Localmente $3K_2$ que se quiera construir. A partir de los elementos del grupo se construyen
conjuntos que generen gr{\a'a}ficas de Cayley localmente $3K_2$ y los filtra usando los valores de los posibles cuellos, finalmente regresa
una lista cuyos elementos son listas con dos entradas, la primera es el
conjunto $T$ que genera una gr{\a'a}fica de Cayley localmente $3K_2$ y la segunda el cuello m{\a'a}s grande que podr{\a'\i}a tener dicha
gr{\a'a}fica. }

 

\subsection{\textcolor{Chapter }{CCGraficaDePuntosYTriangulos}}
\logpage{[ 1, 1, 10 ]}\nobreak
\hyperdef{L}{X86C556EC7C3F2342}{}
{\noindent\textcolor{FuncColor}{$\triangleright$\ \ \texttt{CCGraficaDePuntosYTriangulos({\mdseries\slshape gr{\a'a}fica})\index{CCGraficaDePuntosYTriangulos@\texttt{CCGraficaDePuntosYTriangulos}}
\label{CCGraficaDePuntosYTriangulos}
}\hfill{\scriptsize (function)}}\\


 Recibe una gr{\a'a}fica localmente $3K_2$ y regresa su gr{\a'a}fica bipartita cl{\a'a}nica. }

 

\subsection{\textcolor{Chapter }{CCTsParaCuelloDado}}
\logpage{[ 1, 1, 11 ]}\nobreak
\hyperdef{L}{X7BAD26BB86E655C4}{}
{\noindent\textcolor{FuncColor}{$\triangleright$\ \ \texttt{CCTsParaCuelloDado({\mdseries\slshape grupo, cuello{\textunderscore}de{\textunderscore}tri{\a'a}ngulos, n{\a'u}mero})\index{CCTsParaCuelloDado@\texttt{CCTsParaCuelloDado}}
\label{CCTsParaCuelloDado}
}\hfill{\scriptsize (function)}}\\


 Recibe tres parametros: un grupo, el cuello de triangulos que se desea y el
n{\a'u}mero uno o dos seg{\a'u}n sea el caso de acuerdo al tipo de
gr{\a'a}ficas de Cayley localmente $3K_2$ que se quiera construir. La funci{\a'o}n regresa una lista cuyos elementos son
listas con dos entradas, la primera es el conjunto $T$ que genera una gr{\a'a}fica de Cayley localmente $3K_2$ con el cuello de triangulos que se desea y la segunda es el cuello de su
grafica bipartita cl{\a'a}nica. }

 }

 }

 \def\indexname{Index\logpage{[ "Ind", 0, 0 ]}
\hyperdef{L}{X83A0356F839C696F}{}
}

\cleardoublepage
\phantomsection
\addcontentsline{toc}{chapter}{Index}


\printindex

\newpage
\immediate\write\pagenrlog{["End"], \arabic{page}];}
\immediate\closeout\pagenrlog
\end{document}
